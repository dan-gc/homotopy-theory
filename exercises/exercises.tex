\documentclass{article}

\usepackage[citestyle=authortitle-terse,backref,backend=bibtex]{biblatex}
\addbibresource{homotopy-theory.bib}

\setcounter{secnumdepth}{0}
\usepackage{sectsty}
\sectionfont{\fontsize{13}{20}\selectfont}

\usepackage[left=4cm, right=4cm]{geometry}
\usepackage{palatino,eulervm,dutchcal,xcolor}%fonts
\usepackage{graphicx,subcaption,float}
\usepackage{enumitem,parskip,multicol}
\usepackage{amsthm,amssymb,amsmath,mathtools,thmtools}
\usepackage{tikz,tikz-cd}
\usetikzlibrary{%
	matrix,%
	calc,%
	arrows,%
	shapes,
	decorations.markings,backgrounds,calc,intersections}
\tikzcdset{scale cd/.style={every label/.append style={scale=#1},
			cells={nodes={scale=#1}}}}
\usepackage[bookmarks,bookmarksopen,bookmarksdepth=3]{hyperref}
\hypersetup{%colores
	colorlinks=true,
	urlcolor=blue,
	linkcolor=magenta,
	citecolor=blue,
	filecolor=blue,
	urlbordercolor=white,
	linkbordercolor=white,
	citebordercolor=white,
	filebordercolor=white}
\usepackage{cleveref}

\definecolor{blue-violet}{rgb}{0.54, 0.17, 0.89}
\definecolor{azure}{rgb}{0.0, 0.5, 1.0}
\definecolor{green(ncs)}{rgb}{0.0, 0.62, 0.42}
\definecolor{forestgreen}{rgb}{0.13, 0.55, 0.13}
\definecolor{limegreen}{rgb}{0.2, 0.8, 0.2}
\definecolor{palatinateblue}{rgb}{0.15, 0.23, 0.89}
\definecolor{trueblue}{rgb}{0.0, 0.45, 0.81}
\definecolor{goldenyellow}{rgb}{1.0, 0.87, 0.0}
\definecolor{fashionfuchsia}{rgb}{0.96, 0.0, 0.63}
\definecolor{brightcerulean}{rgb}{0.11, 0.67, 0.84}
\definecolor{jonquil}{rgb}{0.98, 0.85, 0.37}
\definecolor{lavendermagenta}{rgb}{0.93, 0.51, 0.93}
\definecolor{peru}{rgb}{0.8, 0.52, 0.25}
\definecolor{persimmon}{rgb}{0.93, 0.35, 0.0}
\definecolor{persianred}{rgb}{0.8, 0.2, 0.2}
\definecolor{persianblue}{rgb}{0.11, 0.22, 0.73}
\definecolor{persiangreen}{rgb}{0.0, 0.65, 0.58}
\definecolor{persianyellow}{rgb}{0.9, 0.89, 0.0}



\declaretheoremstyle[headfont=\color{trueblue}\normalfont\bfseries,]{colored1}
\declaretheoremstyle[headfont=\color{forestgreen}\normalfont\bfseries,]{colored2}
\declaretheoremstyle[headfont=\color{peru}\normalfont\bfseries,]{colored3}
\declaretheoremstyle[headfont=\color{persiangreen}\normalfont\bfseries,]{colored4}
\declaretheoremstyle[headfont=\color{brightcerulean}\normalfont\bfseries,]{colored5}
\declaretheoremstyle[headfont=\color{lavendermagenta}\normalfont\bfseries,]{colored6}
\declaretheoremstyle[headfont=\color{blue-violet}\normalfont\bfseries,]{colored7}
\declaretheoremstyle[headfont=\color{green(ncs)}\normalfont\bfseries,]{colored8}
\declaretheoremstyle[headfont=\color{peru}\normalfont\bfseries,]{colored9}
\declaretheoremstyle[headfont=\color{persiangreen}\normalfont\bfseries,]{colored10}

\declaretheorem[style=colored1,name=Theorem]{thm}
\declaretheorem[style=colored2,numberlike=thm,name=proposition]{prop}
\declaretheorem[style=colored3,numberlike=thm,name=lemma]{lemma}
\declaretheorem[style=colored4,numberlike=thm,name=corollary]{coro}
\declaretheorem[style=colored5,numbered=no,name=example]{example}
\declaretheorem[style=colored5,numbered=no,name=examples]{exemplos}
\declaretheorem[style=colored6,numbered=no,name=exercise]{exercise}
\declaretheorem[style=colored7,numbered=no,name=remark]{remark}
\declaretheorem[style=colored8,numbered=no,name=claim]{claim}
\declaretheorem[style=colored9,numbered=no,name=definition]{defn}
\declaretheorem[style=colored10,numbered=no,name=question]{question}

\numberwithin{equation}{section}

\newcommand{\A}{\mathbb{A}}
\newcommand{\R}{\mathbb{R}}
\newcommand{\Z}{\mathbb{Z}}
\newcommand{\N}{\mathbb{N}}
\newcommand{\C}{\mathbb{C}}
\newcommand{\Q}{\mathbb{Q}}
\newcommand{\D}{\mathbb{D}}
\renewcommand{\P}{\mathbb{P}}

\newcommand{\Ac}{\mathcal{A}}
\newcommand{\Bc}{\mathcal{B}}
\newcommand{\Cc}{\mathcal{C}}
\newcommand{\Dc}{\mathcal{D}}
\newcommand{\Ec}{\mathcal{E}}
\newcommand{\Fc}{\mathcal{F}}
\newcommand{\Gc}{\mathcal{G}}
\newcommand{\Hc}{\mathcal{H}}
\newcommand{\Ic}{\mathcal{I}}
\newcommand{\Jc}{\mathcal{J}}
\newcommand{\Kc}{\mathcal{K}}
\newcommand{\Lc}{\mathcal{L}}
\newcommand{\Mc}{\mathcal{M}}
\newcommand{\Nc}{\mathcal{N}}
\newcommand{\Oc}{\mathcal{O}}
\newcommand{\Pc}{\mathcal{P}}
\newcommand{\Qc}{\mathcal{Q}}
\newcommand{\Rc}{\mathcal{R}}
\newcommand{\Sc}{\mathcal{S}}
\newcommand{\Tc}{\mathcal{T}}
\newcommand{\Uc}{\mathcal{U}}
\newcommand{\Vc}{\mathcal{V}}
\newcommand{\Wc}{\mathcal{W}}
\newcommand{\Xc}{\mathcal{X}}
\newcommand{\Yc}{\mathcal{Y}}
\newcommand{\Zc}{\mathcal{Z}}

\DeclareMathOperator{\img}{img}
\DeclareMathOperator{\Arg}{Arg}
\DeclareMathOperator{\id}{id}
\DeclareMathOperator{\pt}{pt}
\DeclareMathOperator{\Alt}{Alt}
\DeclareMathOperator{\sgn}{sgn}
\DeclareMathOperator{\hTop}{h-Top}
\DeclareMathOperator{\supp}{supp}
\DeclareMathOperator{\Int}{Int}
\DeclareMathOperator{\Ob}{Ob}
\DeclareMathOperator{\Mor}{Mor}
\DeclareMathOperator{\Top}{Top}
\DeclareMathOperator{\Set}{Set}
\DeclareMathOperator{\CGWH}{CGWH}
\DeclareMathOperator{\Hom}{Hom}
\DeclareMathOperator{\Map}{Map}
\DeclareMathOperator{\Tot}{Tot}
\DeclareMathOperator{\op}{op}
\DeclareMathOperator{\ev}{ev}
\DeclareMathOperator{\hofib}{hofib}
\DeclareMathOperator{\rel}{rel}
\DeclareMathOperator{\Arr}{Arr}
\DeclareMathOperator{\Funct}{Funct}
\DeclareMathOperator{\colim}{colim}

\begin{document}
{\LARGE algebraic topology exercises}
\tableofcontents
\section{some definitions}
\begin{defn}\leavevmode
	\begin{itemize}
		\item An \textbf{\textit{initial object}} in a category $\Cc$ is an object $\varnothing$ such that for any object $x\in\Cc$ there is a unique morphism $\varnothing\to x$ with source $\varnothing$ and target $x$.
		
		\item A \textbf{\textit{pullback}} of the morphisms $f$ and $g$ consists of an object $P$ and two morphisms $p_1:P\to X$ and $p_2:P\to Y$ satisfying the following universal property:
	\[\begin{tikzcd}
		Q\arrow[rrd,bend left,"q_2"]\arrow[ddr,"q_1",swap,bend right]\arrow[dr,dashed,"\phi"]\\
		&P\arrow[r,"p_2"] \arrow[rd, phantom, "\lrcorner", very near start]\arrow[d,"p_1",swap]&Y\arrow[d,"g"]\\
		&X\arrow[r,"f",swap]&Z
	\end{tikzcd}\]
	\item A \textbf{\textit{pushout}} of the morphisms $f$ and $g$ consists of an object $P$ and two morphisms $i_1:P\to X$ and $i_2:P\to Y$ satisfying the following universal property:
	\[\begin{tikzcd}
		Z\arrow[r,"g"]\arrow[d,swap,"f"]&Y\arrow[d,"i_2"]\arrow[ddr,bend left,"j_2"]\\
		X\arrow[r,"i_1",swap]\arrow[drr,bend right,swap,"j_1"]&P\arrow[ul, phantom, "\ulcorner", very near start]\arrow[dr,dashed,"\phi"]\\
		&&Q
	\end{tikzcd}\]
	\item A \textbf{\textit{product}} of $X$ and $Y$ is an object $X\sqcup Y$ and a pair of morphisms $p_1:X\sqcap Y\to X$, $p_2:X\sqcap Y\to Y$ satisfying the following universal property:
	\[\begin{tikzcd}
		Q\arrow[rrd,bend left,"q_2"]\arrow[ddr,"q_1",swap,bend right]\arrow[dr,dashed,"\phi"]\\
		&X\sqcap Y\arrow[r,"p_2"]\arrow[d,"p_1",swap]&Y\\
		&X
	\end{tikzcd}\]
	\item A \textbf{\textit{coproduct}} of $X$ and $Y$ is an object $X\sqcup Y$ and a pair of morphisms $i_1:X\to X\sqcup Y$, $i_2:Y\to X\sqcup Y$ satisfying the following universal property:
	\[\begin{tikzcd}
	&Y\arrow[d,"i_2"]\arrow[ddr,bend left,"j_2"]\\
	X\arrow[r,"i_1",swap]\arrow[drr,bend right,swap,"j_1"]&X\sqcup Y\arrow[dr,dashed,"\phi"]\\
	&&Q
\end{tikzcd}\]
\begin{remark}
	More generally, for $S$ any set and $F:S\to C$ a collection of objects in the category $C$ indexed by $S$, their \textbf{\textit{coproduct}} is an object $\coprod_{s\in S}F(s)$ equipped with maps
	\[F(s)\to\coprod_{s\in S}F(s)\]
	such that this is universal among objects with maps from $F(s)$.
\end{remark}
\item A morphism $i$ has the \textbf{\textit{left lifting property with respect to a morphism $p$}} and $p$ has the \textbf{\textit{right lifting property with respect to $i$}} if for each morphisms $f$ and $g$, if the outer square in the following diagram commutes, there exists $\phi$ (I think not necessarily unique) completing the diagram:
\[\begin{tikzcd}[row sep=large]
	A\arrow[r,"f"]\arrow[d,"i",swap]&X\arrow[d,"p"]\\
	B\arrow[r,"g",swap]\arrow[ur,dashed,"\phi"]&Y
\end{tikzcd}\]

\item For $\Cc$ any category, its \textbf{\textit{arrow category}} $\Arr(\Cc)$ is the category such that
\begin{itemize}
	\item an object $a$ of $\Arr(C)$ is a morphism $a:a_0\to a_1$ of $\Cc$,
	\item a morphism $f:a\to b$ of $\Arr(\Cc)$ is a commutative square
	\[\begin{tikzcd}
		a_0\arrow[r,"f_0"]\arrow[d,swap,"a"]&b_0\arrow[d,"b"]\\
		a_1\arrow[r,swap,"f_1"]&b_1
	\end{tikzcd}\]
	in $\Cc$,
	\item composition in $\Arr(\Cc)$ is given simply by placing commutative squares side by side to get a commutative oblong.
\end{itemize}
This is osomorphic to the functor category
\[\Arr(\Cc):=\Funct(I,C)=[I,C]=C^I\]
for $I$ the interval category $\{0\to 1\}$.
	\end{itemize}
\end{defn}

\section{exercise on mapping cylinder and Hurewicz cofibrations}
\begin{exercise}
	Let $f:X\to Y$ be a map. Let $M_f=X\times[0,1]\cup_fY$ be the \textbf{\textit{mapping cylinder of $f$}}, i.e. the pushout of $X\overset{\cong}{\to}X\times\{0\}\hookrightarrow X\times[0,1]$ and of $f:X\times Y$.
	Let $g:X\to M_f$ be the map $X\overset{\cong}{\to}X\times\{1\}\to M_f$. Let $h:M_f\to Y$ be the map that is induced by $X\times[0,1]\to Y:(x,t)\mapsto f(x)$ and $\id_Y:Y\to Y$. Observe that $f$ is the composition of $g$ and $h$.
	
	In both exercises below you might have to use the fact that pushouts are colimits and that colimits commute with products in $\CGWH$, i.e. $(\colim A_i)\times B$ is canonically homeomorphic with $\colim(A_i \times B)$.
	\begin{enumerate}
		\item Show that $h$ is a deformation retract, and in particular is a homotopy equivalence.
		\item Show that $g : X \to M_f$ is a cofibration. You may use exercise (a), but the direct proof might be simpler.
	\end{enumerate}
\end{exercise}
\begin{proof}[Solution]\leavevmode
	\begin{enumerate}
		\item We have that
		\[\begin{tikzcd}
			X\arrow[r,"f"]\arrow[d,swap,"\id\times 1"]\arrow[rd,bend right,"g"]&Y\arrow[d]\arrow[ddr,bend left,"\id_Y"]\\
			X\times[0,1]\arrow[r,swap]\arrow[drr,bend right,swap,"(x\text{,}t)\mapsto f(x)"]&M_f\arrow[ul, phantom, "\ulcorner", very near start]\arrow[dr,dashed,"h"]\\
			&&Y
		\end{tikzcd}\]
		We must show that there is a homotopy between the identity map on $M_f$ and a retraction from $M_f$ to $Y$. So we want $h:M_f\times[0,1]\to M_f$ such that
		\[h(-,0)=\id_X,\quad \img h(-,1)\subset Y\quad\text{and}\quad h(-,1)|_Y=\id_Y\]
		
		The fact that $h$ is a deformation retract is consequence of this diagram. {\color{red}But I still can't see why it must be a homotopy equivalence…}
		\item Consider the following lifting problem:
		\[\begin{tikzcd}
			X\arrow[d,"g",swap]\arrow[r,"H"]&Y^I\arrow[d,"\pi_0"]\\
			M_f\arrow[ur,dashed]\arrow[r,"h",swap]&Y
		\end{tikzcd}\]
		{\color{red}Looks OK but why should the dashed arrow exist…?}
	\end{enumerate}
\end{proof}
\section{exercise on model categories}
\begin{exercise}[3.1.8 from \cite{riehl}]
	Verify that the class of morphisms $\Lc$ characterized by the left lifting property against a fixed class of morphisms $\Rc$ is closed under coproducts, closed under retracts, and contains the isomorphisms.
\end{exercise}
\begin{proof}[Solution]
	\textbf{\textit{(Coproducts.)}} {\color{magenta} Comment from Sergey: Coproduct of morphisms $A_i\to B_i$ in a category $\Cc$ is the obvious morphism $\sqcup A_i \to \sqcup B_i$. (Because in this construction morphisms $A_i\to B_i$ are seen as objects of what's called the arrow category of the category $\Cc$)}
	
	Suppose the maps $\ell_i:A_i\to B_i$ are in $\Lc$. Then their coproduct in the arrow category is the obvious map $\coprod A_i\to\coprod B_i$.
	
	Explicitly, their coproduct is an arrow $\coprod\ell_i$ and a collection of maps $f_i:\ell_i\to\coprod\ell_i$ such that for any other object $m:A\to B$ in the arrow category and a map $g:\ell\to m$, the following diagram is completed uniquely:
	\[\begin{tikzcd}[scale cd=1.2]
		\ell_i\arrow[r,"f_i"]\arrow[rr,bend right,"g",swap]&\coprod\ell_i\arrow[r,dashed,"\exists!"]&m
	\end{tikzcd}\qquad\forall i\]
	So we conclude that the source of $\coprod\ell_i$ is $\coprod A_i$ and its target $\coprod B_i$. Indeed, we really looking at
	\[\begin{tikzcd}[scale cd=1.2]
		A_i\arrow[r,"\ell_i"]\arrow[d,"f_i^1",swap]&B_i\arrow[d,"f_i^2"]\\
		\coprod A_i\arrow[r,"\coprod\ell_i"]\arrow[d,"\exists!",dashed,swap]&\coprod B_i\arrow[d,dashed,"\exists!"]\\
		A\arrow[r,swap,"m"]&B
	\end{tikzcd}\]
	
	Now consider the following lifting problem with respect to a morphism $r\in\Rc$:
	\[\begin{tikzcd}[row sep=large]
		\coprod A_i\arrow[d,swap,"\coprod\ell_i"]\arrow[r]&\bullet\arrow[d,"r\in\Rc"]\\
		\coprod B_i\arrow[r]&\bullet
	\end{tikzcd}\]
	Since $\ell_i\in\Lc$, we have maps
	\[\begin{tikzcd}[row sep=large]
		A_i\arrow[r]\arrow[d,"\Lc\ni\ell_i",swap]&\coprod A_i\arrow[d]\arrow[r]&\bullet\arrow[d,"r\in\Rc"]\\
		B_i\arrow[r]\arrow[rru,dashed]&\coprod B_i\arrow[r]&\bullet
	\end{tikzcd}\]
	which in turn means we have a unique map
	\[\begin{tikzcd}[row sep=large]
		A_i\arrow[r]\arrow[d,"\Lc\ni\ell_i",swap]&\coprod A_i\arrow[d]\arrow[r]&\bullet\arrow[d,"r\in\Rc"]\\
		B_i\arrow[r]\arrow[rru]&\coprod B_i\arrow[r]\arrow[ur,dashed]&\bullet
	\end{tikzcd}\]
	by the universal property of the coproduct $\coprod B_i$.
	
	To conclude we need to check that the triangles below and above the dashed arrow in the former diagram commute. This follows from the universal property of the coproducts $\coprod A_i$ and $\coprod B_i$ since, \href{https://en.wikipedia.org/wiki/Coproduct#Discussion}{in general},
	\[\Hom\left(\coprod X_i,Y\right)\cong\prod\Hom(X_i,Y).\]
	More explicitly, we now that the paths in red in the following diagrams are the same:
	\[\begin{tikzcd}[row sep=large]
		A_i\arrow[r]\arrow[d,"\Lc\ni\ell_i",swap]&\coprod A_i\arrow[d]\arrow[r]&{\color{red}\bullet}\arrow[d,"r\in\Rc",red]\\
		{\color{red}B_i}\arrow[r,red]\arrow[rru]&{\color{red}\coprod B_i}\arrow[r]\arrow[ur,dashed,red]&{\color{red}\bullet}
	\end{tikzcd}\qquad
		\text{and}\qquad
	\begin{tikzcd}[row sep=large]
		A_i\arrow[r]\arrow[d,"\Lc\ni\ell_i",swap]&\coprod A_i\arrow[d]\arrow[r]&\bullet\arrow[d,"r\in\Rc"]\\
		{\color{red}B_i}\arrow[r,red]\arrow[rru]&{\color{red}\coprod B_i}\arrow[r,red]\arrow[ur,dashed]&{\color{red}\bullet}
	\end{tikzcd}\]
	and also
	\[\begin{tikzcd}[row sep=large]
		{\color{red}A_i}\arrow[r,red]\arrow[d,"\Lc\ni\ell_i",swap]&{\color{red}\coprod A_i}\arrow[d,red]\arrow[r]&{\color{red}\bullet}\arrow[d,"r\in\Rc"]\\
		B_i\arrow[r]\arrow[rru]&{\color{red}\coprod B_i}\arrow[r]\arrow[ur,dashed,red]&\bullet
	\end{tikzcd}\qquad
	\text{and}\qquad
	\begin{tikzcd}[row sep=large]
		{\color{red}A_i}\arrow[r,red]\arrow[d,"\Lc\ni\ell_i",swap]&{\color{red}\coprod A_i}\arrow[d]\arrow[r,red]&{\color{red}\bullet}\arrow[d,"r\in\Rc"]\\
		B_i\arrow[r]\arrow[rru]&\coprod B_i\arrow[r]\arrow[ur,dashed]&\bullet
	\end{tikzcd}\]
	so the conclusion follows from the former comment.
\end{proof}

\section{Hatcher's exercise on Whitehead's theorem}
\begin{thm}[Whitehead, \cite{may}]
	If $X$ is a CW complex and $e:Y\to Z$ is an $n$-equivalence, then $e_*:[X,Y]\to[X,Z]$ is a bijection if $\dim X<n$ and surjection if $\dim X=n$.
\end{thm}
\begin{thm}[Whitehead, \cite{may}]\label{thm:W2}
	An $n$-equivalence between CW complexes of dimension less than $n$ is a homotopy equivalence. A weak equivalence between CW complexes is a homotopy equivalence.
\end{thm}
\begin{thm}[Whitehead (4.5), \cite{hatcher-at}]
	If a map $f:X\to Y$ between connected CW complexes induces isomorphisms $f_*:\pi_n(X)\to\pi_n(Y)$ for all $n$, then $f$ is a homotopy equivalence. In case $f$ is the inclusion of a subcomplex $X\hookrightarrow Y$, the conlusion is stronger: $X$ is a deformation retract of $Y$.
\end{thm}
\begin{exercise}[Hatcher 4.1.12]
	Show that an $n$-connected, $n$-dimensional CW complex is contractible.
\end{exercise}
\begin{proof}[Solution]
	Just recall that $n$-connectedness means that $\pi_i(X)=0$ for all $i\leq n$, which means that $X$ is contractible by \cref{thm:W2}.
\end{proof}
\addcontentsline{toc}{section}{References}
\printbibliography
\clearpage
\end{document}