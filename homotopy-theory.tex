\documentclass{article}
\usepackage{titlesec}

%\usepackage[style=nature,backend=bibtex]{biblatex}
\usepackage[citestyle=authortitle-terse,backref,backend=bibtex]{biblatex}
\addbibresource{homotopy-theory.bib}


\usepackage[left=4cm, right=4cm]{geometry}
\usepackage{palatino}
\usepackage{eulervm}
\usepackage{graphicx}
\usepackage{float}
\usepackage{subcaption}
\usepackage{enumitem}
\usepackage{parskip}
\usepackage{multicol}
\usepackage{amsthm,thmtools,xcolor}
\usepackage{amssymb}
\usepackage{amsmath}
\usepackage{dutchcal}
\usepackage{xcolor}
\usepackage{tikz,tikz-cd}
\tikzcdset{scale cd/.style={every label/.append style={scale=#1},
		cells={nodes={scale=#1}}}}
\usetikzlibrary{%
	matrix,%
	calc,%
	arrows,%
	shapes,
	decorations.markings,backgrounds,calc,intersections
}
\usepackage[bookmarks,bookmarksopen,bookmarksdepth=3]{hyperref}
\hypersetup{%colores
	colorlinks=true,
	urlcolor=blue,
	linkcolor=magenta,
	citecolor=blue,
	filecolor=blue,
	urlbordercolor=white,
	linkbordercolor=white,
	citebordercolor=white,
	filebordercolor=white}
\usepackage{cleveref}
\numberwithin{equation}{section}

\setcounter{secnumdepth}{0}
\usepackage{sectsty}
\sectionfont{\fontsize{13}{20}\selectfont}

\definecolor{blue-violet}{rgb}{0.54, 0.17, 0.89}
\definecolor{azure}{rgb}{0.0, 0.5, 1.0}
\definecolor{green(ncs)}{rgb}{0.0, 0.62, 0.42}
\definecolor{forestgreen}{rgb}{0.13, 0.55, 0.13}
\definecolor{limegreen}{rgb}{0.2, 0.8, 0.2}
\definecolor{palatinateblue}{rgb}{0.15, 0.23, 0.89}
\definecolor{trueblue}{rgb}{0.0, 0.45, 0.81}
\definecolor{goldenyellow}{rgb}{1.0, 0.87, 0.0}
\definecolor{fashionfuchsia}{rgb}{0.96, 0.0, 0.63}
\definecolor{brightcerulean}{rgb}{0.11, 0.67, 0.84}
\definecolor{jonquil}{rgb}{0.98, 0.85, 0.37}
\definecolor{lavendermagenta}{rgb}{0.93, 0.51, 0.93}
\definecolor{peru}{rgb}{0.8, 0.52, 0.25}
\definecolor{persimmon}{rgb}{0.93, 0.35, 0.0}
\definecolor{persianred}{rgb}{0.8, 0.2, 0.2}
\definecolor{persianblue}{rgb}{0.11, 0.22, 0.73}
\definecolor{persiangreen}{rgb}{0.0, 0.65, 0.58}
\definecolor{persianyellow}{rgb}{0.9, 0.89, 0.0}

\declaretheoremstyle[headfont=\color{trueblue}\normalfont\bfseries,]{colored1}
\declaretheoremstyle[headfont=\color{forestgreen}\normalfont\bfseries,]{colored2}
\declaretheoremstyle[headfont=\color{peru}\normalfont\bfseries,]{colored3}
\declaretheoremstyle[headfont=\color{persiangreen}\normalfont\bfseries,]{colored4}
\declaretheoremstyle[headfont=\color{brightcerulean}\normalfont\bfseries,]{colored5}
\declaretheoremstyle[headfont=\color{lavendermagenta}\normalfont\bfseries,]{colored6}
\declaretheoremstyle[headfont=\color{blue-violet}\normalfont\bfseries,]{colored7}
\declaretheoremstyle[headfont=\color{green(ncs)}\normalfont\bfseries,]{colored8}
\declaretheoremstyle[headfont=\color{peru}\normalfont\bfseries,]{colored9}
\declaretheoremstyle[headfont=\color{persiangreen}\normalfont\bfseries,]{colored10}

\declaretheorem[style=colored1,name=Theorem]{thm}
\declaretheorem[style=colored2,numberlike=thm,name=proposition]{prop}
\declaretheorem[style=colored3,numberlike=thm,name=lemma]{lemma}
\declaretheorem[style=colored4,numberlike=thm,name=corollary]{coro}
\declaretheorem[style=colored5,numbered=no,name=example]{example}
\declaretheorem[style=colored5,numbered=no,name=examples]{exemplos}
\declaretheorem[style=colored6,numbered=no,name=exercise]{exercise}
\declaretheorem[style=colored7,numbered=no,name=remark]{remark}
\declaretheorem[style=colored8,numbered=no,name=claim]{claim}
\declaretheorem[style=colored9,numbered=no,name=definition]{defn}
\declaretheorem[style=colored10,numbered=no,name=question]{question}

\newcommand{\A}{\mathbb{A}}
\newcommand{\R}{\mathbb{R}}
\newcommand{\Z}{\mathbb{Z}}
\newcommand{\N}{\mathbb{N}}
\newcommand{\C}{\mathbb{C}}
\newcommand{\Q}{\mathbb{Q}}
\newcommand{\D}{\mathbb{D}}
\renewcommand{\P}{\mathbb{P}}

\newcommand{\Ac}{\mathcal{A}}
\newcommand{\Bc}{\mathcal{B}}
\newcommand{\Cc}{\mathcal{C}}
\newcommand{\Dc}{\mathcal{D}}
\newcommand{\Ec}{\mathcal{E}}
\newcommand{\Fc}{\mathcal{F}}
\newcommand{\Gc}{\mathcal{G}}
\newcommand{\Hc}{\mathcal{H}}
\newcommand{\Ic}{\mathcal{I}}
\newcommand{\Jc}{\mathcal{J}}
\newcommand{\Kc}{\mathcal{K}}
\newcommand{\Lc}{\mathcal{L}}
\newcommand{\Mc}{\mathcal{M}}
\newcommand{\Nc}{\mathcal{N}}
\newcommand{\Oc}{\mathcal{O}}
\newcommand{\Pc}{\mathcal{P}}
\newcommand{\Qc}{\mathcal{Q}}
\newcommand{\Rc}{\mathcal{R}}
\newcommand{\Sc}{\mathcal{S}}
\newcommand{\Tc}{\mathcal{T}}
\newcommand{\Uc}{\mathcal{U}}
\newcommand{\Vc}{\mathcal{V}}
\newcommand{\Wc}{\mathcal{W}}
\newcommand{\Xc}{\mathcal{X}}
\newcommand{\Yc}{\mathcal{Y}}
\newcommand{\Zc}{\mathcal{Z}}

\DeclareMathOperator{\img}{img}
\DeclareMathOperator{\Arg}{Arg}
\DeclareMathOperator{\id}{id}
\DeclareMathOperator{\pt}{pt}
\DeclareMathOperator{\Alt}{Alt}
\DeclareMathOperator{\sgn}{sgn}
\DeclareMathOperator{\hTop}{h-Top}
\DeclareMathOperator{\supp}{supp}
\DeclareMathOperator{\Int}{Int}
\DeclareMathOperator{\Ob}{Ob}
\DeclareMathOperator{\Mor}{Mor}
\DeclareMathOperator{\Top}{Top}
\DeclareMathOperator{\Set}{Set}
\DeclareMathOperator{\CGWH}{CGWH}
\DeclareMathOperator{\Hom}{Hom}
\DeclareMathOperator{\Map}{Map}
\DeclareMathOperator{\Tot}{Tot}
\DeclareMathOperator{\op}{op}
\DeclareMathOperator{\ev}{ev}
\DeclareMathOperator{\hofib}{hofib}
\DeclareMathOperator{\rel}{rel}
\DeclareMathOperator{\coker}{coker}
\DeclareMathOperator{\Arr}{Arr}
\DeclareMathOperator{\Funct}{Funct}
\DeclareMathOperator{\eq}{eq}
\DeclareMathOperator{\coeq}{coeq}

\begin{document}
{\Huge homotopy theory}
\tableofcontents
\section{abstract nonsense}
\begin{defn}\leavevmode
	\begin{itemize}
		\item An \textbf{\textit{initial object}} in a category $\Cc$ is an object $\varnothing$ such that for any object $x\in\Cc$ there is a unique morphism $\varnothing\to x$ with source $\varnothing$ and target $x$.
		
		\item \label{defn:arrow-cat}
		 For $\Cc$ any category, its \textbf{\textit{arrow category}} $\Arr(\Cc)$ is the category such that
		\begin{itemize}
			\item an object $a$ of $\Arr(C)$ is a morphism $a:a_0\to a_1$ of $\Cc$,
			\item a morphism $f:a\to b$ of $\Arr(\Cc)$ is a commutative square
			\[\begin{tikzcd}
				a_0\arrow[r,"f_0"]\arrow[d,swap,"a"]&b_0\arrow[d,"b"]\\
				a_1\arrow[r,swap,"f_1"]&b_1
			\end{tikzcd}\]
			in $\Cc$,
			\item composition in $\Arr(\Cc)$ is given simply by placing commutative squares side by side to get a commutative oblong.
		\end{itemize}
		This is isomorphic to the functor category
		\[\Arr(\Cc):=\Funct(I,C)=[I,C]=C^I\]
		for $I$ the intervale category $\{0\to 1\}$.
		
		\item An \textbf{\textit{equalizer}} is a limit 
		\[\begin{tikzcd}
			\eq\arrow[r,"e"]&X\arrow[r,shift={(0,0.07)},"f"]\arrow[r,shift={(0,-0.07)},"g",swap]&Y
		\end{tikzcd}\]
		over a parallel pair of morphisms $f$ and $g$. This means that for $f:X\to Y$ and $g:X\to Y$ in a category $\Cc$, their equalizer, if it exists, is
		\begin{itemize}
			\item an object $\eq(f,g)\in\Cc$,
			\item a morphism $\eq(f,g)\to X$
			\item such that
			\begin{itemize}
				\item pulled back to $\eq(f,g)$ both morphisms become equal:
				\[\begin{tikzcd}
					\eq(f,g)\arrow[r]&X\arrow[r,"f"]&Y
				\end{tikzcd}\quad=\quad
				[\begin{tikzcd}
					\eq(f,g)\arrow[r]&X\arrow[r,"g"]&Y
				\end{tikzcd}\]
				\item and $\eq(f,g)$ is the universal object with this property.
			\end{itemize}
		\end{itemize}
		The dual concept is that of coequalizer.
		
		\item The concept of coequalizer in a general category is the generalization of the construction where out of two functions $f$ and $g$ between sets $X$ and $Y$ one forms the set $Y/\sim$ of equivalence classes induced by the equivalence relation $f(x)\sim g(y)$. This means the the quotient function $p:Y\to Y/\sim$ satisfies
		\[p\circ f=p\circ g.\]
		In some category $\Cc$, the \textbf{\textit{coequalizer $\coeq(f,g)$}} of two parallel morphisms $f$ and $g$ between two objects $X$ and $Y$, if it exists, is the colimit under the diagram formed by these two morphisms
		\[\begin{tikzcd}
		X\arrow[rr,shift={(0,0.07)},"f"]\arrow[rr,shift={(0,-0.07)},"g",swap]\arrow[dr]&&Y\arrow[ld]\\
		&\coeq(f,g)
		\end{tikzcd}\]
		Equivalently, in a category $\Cc$ a diagram
		\[\begin{tikzcd}
			X\arrow[r,shift={(0,0.07)},"f"]\arrow[r,shift={(0,-0.07)},"g",swap]&Y\arrow[r,"p"]&Z
		\end{tikzcd}\]
		is called a \textbf{\textit{coequalizer}} diagram if
		\begin{enumerate}
			\item $p\circ f=p\circ g$,
			\item $p$ is universal for this property: if $q:Y\to W$ is a morphism of $\Cc$ such that $q\circ f=q\circ g$, then there is a unique morphism $\phi:Z\to W$ such that $\phi\circ p=q$
				\[\begin{tikzcd}
				X\arrow[r,shift={(0,0.07)},"f"]\arrow[r,shift={(0,-0.07)},"g",swap]&Y\arrow[d,"q"]\arrow[r,"p"]&Z\arrow[dl,"\phi",dashed]\\
				&W
			\end{tikzcd}\]
		\end{enumerate}
		The coequalizer in $\Cc$ is equivalently an equializer in the opposite category $\Cc^{\op}$.
		
		\item A \textbf{\textit{pullback}} of the morphisms $f$ and $g$ consists of an object $P$ and two morphisms $p_1:P\to X$ and $p_2:P\to Y$ satisfying the following universal property:
		\[\begin{tikzcd}
			Q\arrow[rrd,bend left,"q_2"]\arrow[ddr,"q_1",swap,bend right]\arrow[dr,dashed,"\phi"]\\
			&P\arrow[r,"p_2"] \arrow[rd, phantom, "\lrcorner", very near start]\arrow[d,"p_1",swap]&Y\arrow[d,"g"]\\
			&X\arrow[r,"f",swap]&Z
		\end{tikzcd}\]
		\item A \textbf{\textit{pushout}} of the morphisms $f$ and $g$ consists of an object $P$ and two morphisms $i_1:P\to X$ and $i_2:P\to Y$ satisfying the following universal property:
		\[\begin{tikzcd}
			Z\arrow[r,"g"]\arrow[d,swap,"f"]&Y\arrow[d,"i_2"]\arrow[ddr,bend left,"j_2"]\\
			X\arrow[r,"i_1",swap]\arrow[drr,bend right,swap,"j_1"]&P\arrow[ul, phantom, "\ulcorner", very near start]\arrow[dr,dashed,"\phi"]\\
			&&Q
		\end{tikzcd}\]
		\begin{remark}
			Other names for the pushout are \textbf{\textit{cofibered product of $X$ and $Y$}} (especially in algebraic categories when $i_1$ and $i_2$ are monomorphisms), or \textbf{\textit{free product of $X$ and $Y$}} with $Z$ \textbf{\textit{amalgamated sum}}, or more simply an \textbf{\textit{amalgamation}} or \textbf{\textit{amalgarm of $X$ and $Y$.}}
		\end{remark}
		\begin{remark}
			If coproducts exist in some category, then the pushout 
			\[\begin{tikzcd}
				Z\arrow[r,"g"]\arrow[d,swap,"f"]&Y\arrow[d,"i_2"]\\
				X\arrow[r,"i_1",swap]&X\coprod_Z Y\arrow[ul, phantom, "\ulcorner", very near start]
			\end{tikzcd}\]
			is equivalently the coequalizer 
			\[\begin{tikzcd}
				X\arrow[r,shift={(0,0.07)},"i_1\circ f"]\arrow[r,shift={(0,-0.07)},"i_2\circ g",swap]&X\coprod Y\arrow[r]&X\coprod_Z Y
			\end{tikzcd}\]
			of the two morphisms induced by $f$ and $g$ into the coproduct of $X$ with $Y$.
		\end{remark}
		\begin{example}[\href{https://en.wikipedia.org/wiki/Pushout_(category_theory)}{wiki}]\leavevmode
			\begin{itemize}
				\item If $X$, $Y$ and $Z$ are sets and $f,g$ are functions, the pushout of $f$ and $g$ is the disjoint union of $X$ and $Y$ where elements sharing a common preimage in $Z$ are identified, i.e. $P=\left(X\coprod Y\right)/\sim$ where $\sim$ is the finest equivalence relation such that $f(z)\sim g(z)$ for all $z\in Z$.
			
			In particular, if $X$ and $Y$ are subsets of some larger set $W$ and $Z$ is their intersection, with $f$ and $g$ the inclusion maps of $Z$ into $X$ and $Y$, then the pusout can be canonically identified with the union $X\cup Y\subseteq W$.
			\item \label{adjuntion-space}The construcion of \textbf{\textit{adjuntion spaces}} is an example of pushouts in $\Top$. More precisely, if $Z$ is a subspace of $Y$ and $g:Z\to Y$ is the inclusion map, we can glue $Y$ to another space $X$ along $Z$ using an \textbf{\textit{attaching map}} $f:Z\to X$. The result is the \textbf{\textit{adjuntion space}} $X\cup_f Y$ which is just the pushout of $f$ and $g$. More generally, all identification spaces may be regarded as pushouts in this way. See \cref{defn:CW-complex}.
			\end{itemize}
		\end{example}
		\item A \textbf{\textit{product}} of $X$ and $Y$ is an object $X\sqcup Y$ and a pair of morphisms $p_1:X\sqcap Y\to X$, $p_2:X\sqcap Y\to Y$ satisfying the following universal property:
		\[\begin{tikzcd}
			Q\arrow[rrd,bend left,"q_2"]\arrow[ddr,"q_1",swap,bend right]\arrow[dr,dashed,"\phi"]\\
			&X\sqcap Y\arrow[r,"p_2"]\arrow[d,"p_1",swap]&Y\\
			&X
		\end{tikzcd}\]
		\item A \textbf{\textit{coproduct}} of $X$ and $Y$ is an object $X\sqcup Y$ and a pair of morphisms $i_1:X\to X\sqcup Y$, $i_2:Y\to X\sqcup Y$ satisfying the following universal property:
		\[\begin{tikzcd}
			&Y\arrow[d,"i_2"]\arrow[ddr,bend left,"j_2"]\\
			X\arrow[r,"i_1",swap]\arrow[drr,bend right,swap,"j_1"]&X\sqcup Y\arrow[dr,dashed,"\phi"]\\
			&&Q
		\end{tikzcd}\]
		\begin{remark}
			More generally, for $S$ any set and $F:S\to\Cc$ a collection of objects in $\Cc$ indexed by $S$, their \textbf{\textit{coproduct}} is an object
			\[\coprod_{s\in S}F(s)\]
			equipped with maps
			\[F(s)\to\coprod_{s\in S}F(s)\]
			such that this is universal among objects with maps from $F(s)$.
		\end{remark}
		\item A morphism $i$ has the \textbf{\textit{left lifting property with respect to a morphism $p$}} and $p$ has the \textbf{\textit{right lifting property with respect to $i$}} if for each morphisms $f$ and $g$, if the outer square in the following diagram commutes, there exists $\phi$ (I think not necessarily unique) completing the diagram:
		\[\begin{tikzcd}[row sep=large]
			A\arrow[r,"f"]\arrow[d,"i",swap]&X\arrow[d,"p"]\\
			B\arrow[r,"g",swap]\arrow[ur,dashed,"\phi"]&Y
		\end{tikzcd}\]
		\item The \textbf{\textit{kernel}} of a morphism is that part of its domain which is sent to zero. Formally, in a category with an initial object 0 and pullbacks, the \textbf{\textit{kernel $\ker f$}} of a morphism $f:A\to B$ is the pullback $\ker(f)\to A$ along $f$ of the unique morphism $0\to B$
		
		More explicitly, this characterizes the object $\ker(f)$ as \textit{the} object (unique up to isomorphism) that satisfies the following universal property:
		\begin{quote}
			for every object $C$ and every morphism $h:C\to A$ such that $f\circ h=0$ is the zero morphism, there is a unique morphism $\phi:C\to\ker(f)$ such that $h=p\circ\phi$.
		\end{quote}
		\[\begin{tikzcd}
			C\arrow[dr,dashed,"\phi"]\arrow[ddr,bend right,swap,"h"]\\
			&\ker(f)\arrow[d]\arrow[r]&0\arrow[d]\\
			&A\arrow[r,"f",swap]&B\arrow[ul, phantom, "\ulcorner", very near start]
		\end{tikzcd}\]
		\item In a category with a terminal object 1, the \textbf{\textit{cokernel}} of a morphism $f:A\to B$ is the pushout (arrows $h$ and $\phi$ apply if terminal object is zero)
		\[\coker(f):=1\sqcup_AB\qquad\qquad\begin{tikzcd}
			A\arrow[r,"f"]\arrow[d]\arrow[dr, phantom, "\lrcorner", very near start]&B\arrow[d]\arrow[rdd,bend left,"h"]\\
			1\arrow[r]&\coker(f)\arrow[dr,dashed,"\phi"]\\
			&&C
		\end{tikzcd}\]
		In the case when the terminal object is in fact zero object, one can, more explicitly, characterize the object $\coker(f)$ with the following universal property:
		\begin{quote}
			for every object $C$ and every morphism $h:B\to C$ such that $h\circ f=0$ is the zero morphism, there is a unique morphism $\phi:\coker(f)\to C$ such that $h=\phi\circ i$.
		\end{quote}
		
		\item A morphism $f:X\to Y$ is a \textbf{\textit{monomorphism}} if for every object $Z$ and every pair of morphisms $g_1,g_2:Z\to X$ then
		\[f\circ g_1=f\circ g_2\implies g_1=g_2.\]
		\[\begin{tikzcd}
			Z\arrow[r,shift={(0,.06)},"g_1"]\arrow[r,swap,shift={(0,-.06)},"g_2"]\arrow[rr,bend left,"f\circ g_1"]\arrow[rr,bend right,swap,"f\circ g_2"]&X\arrow[r,"f"]&Y
		\end{tikzcd}\]
		Equivalently, $f$ is a monomorphism if for every $Z$ the hom-functor $\Hom(Z,-)$ takes it to an injective function
		\[\begin{tikzcd}
			\Hom(Z,X)\arrow[r,hook,"f_*"]&\Hom(Z,Y).
		\end{tikzcd}\]
		Being a monomorphism in a category $\Cc$ means equivalently that it is an epimorphism in the opposite category $\Cc^{\op}$.
		
		\item A morphism $f:X\to Y$ is a \textbf{\textit{epimorphism}} if for every object $Z$ and every pair of morphisms $g_1,g_2:Y\to Z$ then
		\[g_1\circ f=g_2\circ f\implies g_1=g_2.\]
		\[\begin{tikzcd}
			X\arrow[r,"f"]\arrow[rr,bend right,swap,"g_2\circ f"]\arrow[rr,bend left,"g_1\circ f"]&Y\arrow[r,shift={(0,.06)},"g_1"]\arrow[r,swap,shift={(0,-.06)},"g_2"]&Z
		\end{tikzcd}\]
		Equivalently, $f$ is a epimorphism if for every $Z$ the hom-functor $\Hom(-,Z)$ takes it to an injective function
		\[\begin{tikzcd}
			\Hom(Y,Z)\arrow[r,hook,"f^*"]&\Hom(X,Z).
		\end{tikzcd}\]
		Being a monomorphism in a category $\Cc$ means equivalently that it is an monomorphism in the opposite category $\Cc^{\op}$.
	\end{itemize}
\end{defn}
\section{elementary concepts}
\begin{defn}\leavevmode
	\begin{itemize}
		\item Let $X$ and $Y$ be topological spaces and $f,g:X\to Y$ continuous maps. An \textbf{\textit{homotopy}} from $f$ to $g$ is a continuous map
		\[H:X\times[0,1]\to Y,\qquad(x,t)\mapsto H(x,t)=H_t(x)\])
		such that $f(x)=H(x,0)$ and $g(x)=H(x,1)$ for all $x\in X$. We denote this situation by $f\simeq g$. The homotopy relation $\simeq$ is an equivalence relation on the set of continuous maps $X\to Y$. A homotopy of maps $H_t:X\to Y$ is called \textbf{\textit{relative to $A\subset X$}} if $H_t|_A$ is constant.
		
		\item Topological spaces and homotopy classes of maps form a quotient category of $\Top$, the \textbf{\textit{homotopy category $\hTop$}}, where comoposition of homotopy classes is induced by composition of representing maps. If $f:X\to Y$ represents an isomorphism in $\hTop$, then $f$ is called a \textbf{\textit{homotopy equivalence}} or \textbf{\textit{$\operatorname{h}$-equivalence}}. In explicit termins this means $f:X\to Y$ is a homotopy equivalence if there exists $g:Y\to X$, a \textbf{\textit{homotopy inverse of $f$}}, such that $gf$ and $fg$ are both homotopic to the identity. Spaces $X$ and $Y$ are called \textbf{\textit{homotopy equivalent}} or of the same \textbf{\textit{homotopy type}} if there exists a homotopy equivalence $X\to Y$. A space is \textbf{\textit{contractible}} if it is homotopy equivalent to a point. A map $f:X\to Y$ is \textbf{\textit{null homotopic}} if it is homotopic to a constant map.
		
		\item Let $(X,x_0)$ be a pointed topological space and $s_0\in S^n$. The elements of the \textbf{\textit{$n$-th homotopy group}} are homotopy classes of maps $(S^n,s_0)\to (X,x_0)$. Equivalently, they are homotopy classes of maps $(I^n,\partial I^n)\to (X,x_0)$. (Homotopies are required to preserve the base points, $s_0\mapsto x_0$ or $\partial I^n\mapsto x_0$.)
		
		Also,
		\[\pi_n(X,*)=[(I^n,\partial I^n),(X,\{*\})]\cong[I^n/\partial I^n,X]^0\]
		where $[X,Y]$ denotes the set of homotopy classes $[f]$ of maps $[f]:X\to Y$.
		\begin{prop}
			$\pi_n(X,x_0)$ is an abelian group for all $n\in\N$.
		\end{prop}
				
		\item Let $A$ be a subspace of $X$ and $x_0\in A$. The elements of the \textbf{\textit{relative homotopy group $\pi_n(X,A,x_0)$}} are homotopy classes of maps $(I^n,\partial I^n,J^{n-1})\to (X,A,x_0)$ where $J^{n-1}$ is the union of all but one face of $I^n$. That is,
		\[\pi_{n+1}(X,A,*)=[(I^{n+1},\partial I^{n+1},J^n),(X,A,x_0)].\]
		
		The elements of such a group are homotopy classes of based maps $D^n\to X$ which carry the boundary $S^{n-1}$ into $A$. Two maps $f,g$ are called \textbf{\textit{homotopic relative to $A$}} if they are homotopic by a basepoint-preserving homotopy $F:D_n\times[0,1]\to X$ such that, for each $p$ in $S^{n-1}$ and $t$ in $[0,1]$, the element $F(p,t)$ is in $A$. Ordinary homotopy groups are recovered for the case in which $A=\{x_0\}$.
		\begin{remark}
			This construction is motivated by looking for the kernel of the induced map $i_*:\pi_n(A,x_0)\to\pi_n(X,x_0)$ by the inclusion. This map is in general not injective, and the kernel consists of ?
		\end{remark}
		\item For any pair $(X,A,x)$ we have a long exact sequence
		\[\begin{tikzcd}[column sep=small]
			\pi_n(A,x_0)\arrow[r,"i_*"]&\pi_{n}(X,x_0)\arrow[r,"j_*"]&\pi_{n-1}(A,x_0)\arrow[r,"\partial"]&\pi_{n-1}(X,x_0)\arrow[r]&\cdots\arrow[r]&\pi_0(X,x_0)
		\end{tikzcd}\]
		where $i$ and $j$ are the inclusions $(A,x_0)\hookrightarrow(X,x_0)$ and $(X,x_0,x_0)\hookrightarrow(X,A,x_0)$. The map $\partial$ comes from restricting maps $(I^n,\partial I^n,J^{n-1})\to (X,A,x_0)$ to $I^{n-1}$, or by restricting maps $(D^n,S^{n-1},s_0)\to (X,A,x_0)$. The map, called the \textbf{\textit{boundary map}}, is a homomorphism when $n>1$.
		
		\item A space $X$ with basepoint $x_0$ is called \textbf{\textit{$n$-connected}} if $\pi_i(X,x_0)=0$ for $i\leq n$. Thus 0-connected means path-connected and 1 connected means simply-connected.
		
		\item A pair $(X,A)$ is \textbf{\textit{$n$-connected}} if $\pi(X,A,x_0)=0$ for $i\leq n$.
		
		\item Two pointed spaces $(X,x_0)$ and $(Y,y_0)$ are \textbf{\textit{$n$-equivalent}} if $\pi_i(X,x_0)\cong\pi_i(Y,y_0)$ for all $i\leq n$.
	\end{itemize}
\end{defn}

\section{the right category}
\begin{itemize}
	\item We don't care so much about $\Top$. We care much more about $\CGWH$, the full subcategory of $\Top$ on \textbf{\textit{compactly generated wakly Hausdorff}} spaces.
	\item $X$ is \textbf{\textit{compactly generated}} if, for any subset $C\subset X$, and for all continuous maps $f:K\to X$ from compact Housdorff spaces, \[\text{if } f^{-1}(C) \text{is closed in }K\text{, then } C\text{ is closed}.\]
	\begin{claim}[What I picked up from the lecture]
		If $X$ is compactly generated, then $X$ is weakly Hausdorff if the diagonal subset $\Delta_X\subset X\times X$ is {\color{orange}$k$-closed}.
	\end{claim}
	From \cite{may}: The ordinary category of spaces allows pathology that obstructs a clean development of the foundations. The homotopy and homology groups of spaces are supported on compact subspaces, and it turns out that if one assumes a separation property that is a little weaker than the Hausdorff property, then one can refine the point-set topology of spaces to eliminate such pathology without changing these invariants.
	
	One major source of point-set level pathology can be passage to quotient spaces. Use of compactly generated topologies alleviates this.
	\begin{prop}
		If $X$ is compactly generated and $\pi:X\to Y$ is a quotient map, then $Y$ is compactly generated if and only if $(\pi\times \pi)^{-1}(\Delta Y)$ is closed in $X\times X$
	\end{prop}
	The interpretation is that a quotient space of a compactly generated space by a “closed equivalence relation” is compactly generated.
	
	{\color{cyan}Several other propositions follow in \cite{may}. Now some other notes from the lectures:}
	
	In $\CGWH$, $\Hom(X,Y)$ is a space with the compact-open topology. {\color{orange} This is a compactly generated space, $\mathbf{k}(\Hom(X,Y))$}. 
	\begin{align*}
		\Map(X,Y):=\text{ the space of maps }X\to Y.\\
		\Map(X\times Y,Z)\cong\Map(X,\Map(Y,Z))\\
		\Hom(X\times Y,Z)\cong \Hom(X,\Map(Y,Z))
	\end{align*}
	In the last line, product is product in $\CGWH$, not in $\Top$.
	
	The functor $-\times Y$ is left adjoint to $\Map(Y,-)$.
\end{itemize}
\section{cofibrations}
\begin{itemize}
	\item A \textbf{\textit{homotopy}} $X\times I\to Y$ is the same as a map $X\to \Map(I,Y)$.
	\item A map $A\to X$ is a \textbf{\textit{Hurewicz cofibration}} for any $g:X\to Y$ and any homotopy $H:A\times I\to Y$ such that \[\begin{tikzcd}
		A\times\{0\}\arrow[r]\arrow[d]&A\times I\arrow[d]\\
		X\arrow[r,"g",swap]&Y
	\end{tikzcd}\]
	there is $H:X\times I\to Y$,
	\[\begin{tikzcd}
		X\times\{0\}\arrow[r]\arrow[d,"g"]&A\times I\arrow[d]\\
		X\times I\arrow[r,"H'",swap]&Y
	\end{tikzcd}\]
	
	\[\begin{tikzcd}
		A\times I\arrow[dr,"H"]\arrow[d]\\
		X\times I\arrow[r,"H'",swap]&Y
	\end{tikzcd}\]
\end{itemize}
\begin{example}
$\partial D^n\to D$ is a Huerwicz cofibration. {\color{orange} Why?}
\end{example}
\section{model structures}
\begin{defn}[\cite{riehl}]
	A \textbf{\textit{weak factorization system $(\Lc,\Rc)$}} on a category $\Mc$ is comprised o two clases of morphisms $\Lc$ and $\Rc$ so that
	\begin{enumerate}
		\item Every morphism in $\Mc$ may be factored as a morphism in $\Lc$ followed by a morphism in $\Rc$:
		\[\begin{tikzcd}
			\bullet\arrow[rr,"f"]\arrow[rd,"\Lc\ni\ell",swap]&&\bullet\\
			&\bullet\arrow[ru,"r\in\Rc"]
		\end{tikzcd}\]
		\item The maps in $\Lc$ have the \textbf{\textit{left lifting property}} with respect to each map in $\Rc$ and equivalently the maps in $\Rc$ have the \textbf{\textit{right lifting property}} with respect to each map in $\Lc$, that is, any commutative square
		\[\begin{tikzcd}
			\bullet\arrow[d,"\Lc\ni\ell",swap]\arrow[r]&\bullet\arrow[d,"r\in\Rc"]\\
			\bullet\arrow[ur,dashed]\arrow[r]&\bullet
		\end{tikzcd}\]
		admits a diagonal filler as indicated making both triangles commute.
		\item The classes $\Lc$ and $\Rc$ are each closed under retracts in the arrow category: given a commutative diagram
		\[\begin{tikzcd}
			\bullet\arrow[r]\arrow[d,"t",swap]\arrow[rr,bend left,no head,shift left=.3]\arrow[rr,bend left,no head,shift right=.3]&\bullet\arrow[r]\arrow[d,"s"]&\bullet\arrow[d,"t"]\\
			\bullet\arrow[r]\arrow[rr,bend right,no head, shift right=.3]\arrow[rr,bend right,no head, shift left=.3]&\bullet\arrow[r]&\bullet
		\end{tikzcd}\]
		if $s$ is in that class then so is its retract $t$.
	\end{enumerate}
\end{defn}
\begin{defn}[Lecture]
A \textbf{\textit{model structure}} on a category $\Ac$ is a choice of subcategories $\Wc,\Cc,\Fc$ called \textbf{\textit{weak-equivalences}}, \textbf{\textit{cofibrations}} and \textbf{\textit{fibrations}} with the following properties:
\begin{enumerate}
	\item Given $\bullet\overset{f}{\to}\bullet\overset{g}{\to}\bullet$, if either 2 out of 3 among $f,g,f\circ g$ are in $\Wc$ then all of them are.
	\item $(\Cc\cap\Wc,\Fc)$ and $(\Cc,\Fc\cap\Wc)$ are both weak factorization systems.
	$(\Bc,\Dc)$ is a weak factorization system.
	\begin{enumerate}
		\item Any morphism in $\Ac$ can be factored as a morphism in $\Bc$ followed by a morphism in $\Dc$.
		\item Lifts:
		\[\begin{tikzcd}[scale cd=1.2]
			\bullet\arrow["f",d,swap]\arrow[r]&\bullet\arrow[d,"g"]\\
			\bullet\arrow[ur,dashed,"\exists"]\arrow[r]&\bullet
		\end{tikzcd}\]
	\end{enumerate}
\end{enumerate}
\end{defn}
Two interesting model category structures on $\CGWH$.
\begin{enumerate}
\item Hurewicz model structure (Strom).
\begin{itemize}
	\item Cofibrations:= Huerwicz cofibrations.
	\item Fibrations:= maps $E\to B$ such that for all spaces $X$ [Photo1].
	\item Weak equivalences:= homotopy equivalences.
\end{itemize}
\item Quillen model structure.
\begin{itemize}
	\item Cofibrations = retracts of relative cell complexes.
	\item (Serre) Fibrations = \begin{tikzcd}
		D^n\arrow[r]\arrow[d]&E\arrow[d]\\
		D^n\times I\arrow[ur,dashed]\arrow[r]&B
	\end{tikzcd}
	\item Weak equivalences: $f:X\to Y$
\end{itemize}
\end{enumerate}

\begin{exercise}[3.1.8 from \cite{riehl}]
Verify that the class of morphisms $\Lc$ characterized by the left lifting property against a fixed class of morphisms $\Rc$ is closed under coproducts, closed under retracts, and contains the isomorphisms.
\end{exercise}
\begin{proof}[Solution]
\textbf{\textit{(Coproducts.)}} {\color{magenta} Sergey: Coproduct of morphisms $A_i\to B_i$ in a category $\Cc$ is the obvious morphism $\sqcup A_i \to \sqcup B_i$. (Because in this construction morphisms $A_i\to B_i$ are seen as objects of what's called the arrow category of the category $\Cc$)}

Suppose the maps $\ell_i:A_i\to B_i$ are in $\Lc$. Then their coproduct in the arrow category is the obvious map $\coprod A_i\to\coprod B_i$.

Explicitly, their coproduct is an arrow $\coprod\ell_i$ and a collection of maps $f_i:\ell_i\to\coprod\ell_i$ such that for any other object $m:A\to B$ in the arrow category and a map $g:\ell\to m$, the following diagram is completed uniquely:
\[\begin{tikzcd}[scale cd=1.2]
	\ell_i\arrow[r,"f_i"]\arrow[rr,bend right,"g",swap]&\coprod\ell_i\arrow[r,dashed,"\exists!"]&m
\end{tikzcd}\qquad\forall i\]
So we conclude that the source of $\coprod\ell_i$ is $\coprod A_i$ and its target $\coprod B_i$. Indeed, we really looking at
\[\begin{tikzcd}[scale cd=1.2]
	A_i\arrow[r,"\ell_i"]\arrow[d,"f_i^1",swap]&B_i\arrow[d,"f_i^2"]\\
	\coprod A_i\arrow[r,"\coprod\ell_i"]\arrow[d,"\exists!",dashed,swap]&\coprod B_i\arrow[d,dashed,"\exists!"]\\
	A\arrow[r,swap,"m"]&B
\end{tikzcd}\]

Now consider the following lifting problem with respect to a morphism $r\in\Rc$:
\[\begin{tikzcd}[row sep=large]
	\coprod A_i\arrow[d,swap,"\coprod\ell_i"]\arrow[r]&\bullet\arrow[d,"r\in\Rc"]\\
	\coprod B_i\arrow[r]&\bullet
\end{tikzcd}\]
Since $\ell_i\in\Lc$, we have maps
\[\begin{tikzcd}[row sep=large]
	A_i\arrow[r]\arrow[d,"\Lc\ni\ell_i",swap]&\coprod A_i\arrow[d]\arrow[r]&\bullet\arrow[d,"r\in\Rc"]\\
	B_i\arrow[r]\arrow[rru,dashed]&\coprod B_i\arrow[r]&\bullet
\end{tikzcd}\]
which in turn means we have unique maps
\[\begin{tikzcd}[row sep=large]
	A_i\arrow[r]\arrow[d,"\Lc\ni\ell_i",swap]&\coprod A_i\arrow[d]\arrow[r]&\bullet\arrow[d,"r\in\Rc"]\\
	B_i\arrow[r]\arrow[rru]&\coprod B_i\arrow[r]\arrow[ur,dashed]&\bullet
\end{tikzcd}\]
by the universal property of the coproduct $\coprod B_i$.

So, to check that the lower-right triangle commutes, it would be sufficient to show that the map $B_i\to\coprod B_i$ "can be cancelled" since
\[\begin{tikzcd}[row sep=large]
	A_i\arrow[r]\arrow[d,"\Lc\ni\ell_i",swap]&\coprod A_i\arrow[d]\arrow[r]&{\color{red}\bullet}\arrow[d,"r\in\Rc",red]\\
	{\color{red}B_i}\arrow[r,red]\arrow[rru]&{\color{red}\coprod B_i}\arrow[r]\arrow[ur,dashed,red]&{\color{red}\bullet}
\end{tikzcd}\quad
\begin{array}{c}
	\text{is already}\\
	\text{the same as}
\end{array}\quad
\begin{tikzcd}[row sep=large]
A_i\arrow[r]\arrow[d,"\Lc\ni\ell_i",swap]&\coprod A_i\arrow[d]\arrow[r]&\bullet\arrow[d,"r\in\Rc"]\\
{\color{red}B_i}\arrow[r,red]\arrow[rru]&{\color{red}\coprod B_i}\arrow[r,red]\arrow[ur,dashed]&{\color{red}\bullet}
\end{tikzcd}\]
Likeways, to make sure that the remaining triangle commutes we observe that
\[\begin{tikzcd}[row sep=large]
	{\color{red}A_i}\arrow[r,red]\arrow[d,"\Lc\ni\ell_i",swap]&{\color{red}\coprod A_i}\arrow[d,red]\arrow[r]&{\color{red}\bullet}\arrow[d,"r\in\Rc"]\\
	B_i\arrow[r]\arrow[rru]&{\color{red}\coprod B_i}\arrow[r]\arrow[ur,dashed,red]&\bullet
\end{tikzcd}\quad
\begin{array}{c}
	\text{is already}\\
	\text{the same as}
\end{array}
\quad
\begin{tikzcd}[row sep=large]
	{\color{red}A_i}\arrow[r,red]\arrow[d,"\Lc\ni\ell_i",swap]&{\color{red}\coprod A_i}\arrow[d]\arrow[r,red]&{\color{red}\bullet}\arrow[d,"r\in\Rc"]\\
	B_i\arrow[r]\arrow[rru]&\coprod B_i\arrow[r]\arrow[ur,dashed]&\bullet
\end{tikzcd}\]
{\color{magenta}Why can we "cancel" the maps $A_i\to\coprod A_i$ and $B_i\to\coprod B_i$?}
\end{proof}

\begin{remark}[Plan]
	Blakers-Massey excision theorem (relies on technical lema, proof from Tom Dieck's book) $\implies$ Cellular approximation. Also $\implies$ Freudental theorem.
\end{remark}
\begin{exercise}
	$X\to M_f\to Y$. Prove $X\to M_f$ is a cofibration.
\end{exercise}
\section{whitehead theorem}
\iffalse\begin{defn}
	The appropriate analogue of the Cartesian product in the category of based spaces is the \textbf{\textit{smash product}} $X\wedge Y$ defined by
	\[X\wedge Y=X\times Y/X\vee Y.\]
	Here $X\vee Y$ is viewed as the subspace of $X\times Y$ consisting of those pairs $(x,y)$ such that either $x$ is the basepoint of $X$ or $y$ is the basepoint of $Y$.
\end{defn}\fi
We introduce a large class of spaces, called CW complexes, between which a weak equivalence is necessarily a homotopy equivalence. Thus, for such spaces, the homotopy groups are, in a sense, a complete set of invariants. Moreover, we shall see that every space is weakly equivalent to a CW complex.
\begin{defn}[\cite{may}]\label{defn:CW-complex}\leavevmode
	\begin{enumerate}
		\item A \textbf{\textit{CW complex}} $X$ is a space $X$ which is the union of an expanding sequence of subspaces $X^n$ such that, inductively, $X^0$ is a discrete set of points (called \textbf{\textit{vertices}}) and $X^{n+1}$ is the pushout obtained from $X^n$ by attaching disks $D^{n+1}$ along \textbf{\textit{attaching maps}} $j:S^n\to X^n$. {\color{cyan}Thus $X^{n+1}$ is the quotient space obtained from $X^n\cup(J_{n+1}\times D^{n+1})$ by identifying $(j,x)$ with $j(x)$ for $x\in S^n$, where $J_{n+1}$ is the discrete set of such attaching maps $j$ (see \cref{adjuntion-space})}. Each resulting map $D^{n+1}\to X$ is called a \textbf{\textit{cell}}. The subspace $X^n$ is called the \textbf{\textit{$n$-skeleton}} of $X$.
			\[\begin{tikzcd}
			S^n\arrow[r,"i",hook]\arrow[d,swap,"j"]&D^{n+1}\arrow[d,]\\
			X^n\arrow[r]&X^{n+1}\arrow[ul, phantom, "\ulcorner", very near start]
		\end{tikzcd}\]
	\end{enumerate}
\end{defn}
\begin{lemma}[HELP]
	content...
\end{lemma}
\begin{thm}[Whitehead, \cite{may}]
	If $X$ is a CW complex and $e:Y\to Z$ is an $n$-equivalence, then $e_*:[X,Y]\to[X,Z]$ is a bijection if $\dim X<n$ and surjection if $\dim X=n$.
\end{thm}
\begin{thm}[Whitehead, \cite{may}]\label{thm:W2}
	An $n$-equivalence between CW complexes of dimension less than $n$ is a homotopy equivalence. A weak equivalence between CW complexes is a homotopy equivalence.
\end{thm}
\begin{thm}[Whitehead (4.5), \cite{hatcher-at}]
	If a map $f:X\to Y$ between connected CW complexes induces isomorphisms $f_*:\pi_n(X)\to\pi_n(Y)$ for all $n$, then $f$ is a homotopy equivalence. In case $f$ is the inclusion of a subcomplex $X\hookrightarrow Y$, the conlusion is stronger: $X$ is a deformation retract of $Y$.
\end{thm}
\begin{exercise}[Hatcher 4.1.12]
	Show that an $n$-connected, $n$-dimensional CW complex is contractible.
\end{exercise}
\begin{proof}[Solution]
	Just recall that $n$-connectedness means that $\pi_i(X)=0$ for all $i\leq n$, which means that $X$ is contractible by \cref{thm:W2}.
\end{proof}
\section{lecture notes}
\subsection{14 mar}

\[(X^Y)^Z\cong Z^{Y\times X}\]

\begin{align*}
	g:X'\to X\\\\
	\Hom(X,Y)\mapsto\Hom(X',Y)
\end{align*}

\begin{align*}
	\Hom(A,B)&\cong\Hom(A,B')\text{ natual in }A\implies\\ \Hom(B,B)\cong\Hom(B,B')&\& \Hom(B',B)\cong\Hom(B',B')\\\implies B&\cong B'.
\end{align*}


\begin{itemize}
	\item for $(\impliedby)$ commutativity of the hypotesis gives us commutativiuty of the rightmost square in the diagram below. In fact, the double square diagram below is a rephrasing of the hypothesis.
	
	\item Lemma 2. To build CW complexes
	
	\item {\color{cyan} Some good concepts are pushouts, coproducts, direct limits.}
	\item What we did? Prove the bijection between the homotopic sets given an $n$-equivalence.
	\item Defined smash.
	\item $\pi_n$ of loop space is the same as $\pi_{n+1}$ of original space.
	\item Then we moved on to homotopic pushouts and pullback. We saw, for instance, that if in a double square diagram each of the squares is a homotopic pushout, then so is the outer square.
	\item We also looked at those exact sequences on cofibers, spaces of homotopy classes, cohomology and (barely) loop spaces. There was a lemma about this.
	\item Next time: cofiber of cofiber is homotopy equivalence, then fibers, fibrations and probably *some name* theorem.
\end{itemize}

\subsection{18 mar}
%Recall some basic definitions:

%\begin{defn}
%	A map $p:E\to B$ has the \textbf{\textit{homotopy lifting property (HLP)}} for the space $X$ if the following holds: for each homotopy $h:W\times I\to B$ and each map $a:X\to E$
%\end{defn}
\begin{lemma}[Yoneda]	
	\[\left\{\text{Natural transformations }\Hom(-,X)\to F\right\}\cong F(X)\]
\end{lemma}
\begin{coro}
	$(\Hom(-,X)\to\Hom(-,Y))\cong\Hom(X,Y)$.
\end{coro}
\begin{coro}
	The correspondence $X\mapsto\Hom(-,X)$ is fully faithful, that is, the correspondence $\Hom(X,X')\to\Hom(\Hom(-,X),\Hom(-,X'))$ is injective and bijective. (The right hand side are natural transformations of functors.)
\end{coro}
\begin{proof}[Solution of exercise 1] The latter correspondence sends isomorphisms to isomorphisms. Since we are given a natural isomorphism in the problem, we conclude $X\cong X'$.
\end{proof}

\begin{lemma}
	Let $E\times_BX$ be the pullback of
	\[\begin{tikzcd}
		&E\arrow[d,two heads]\\
		X\arrow[r,"\simeq"]&B
	\end{tikzcd}\]
	be such that $E\to B$ is an homotopy fibration and $f:X\to B$ is a homotopy equivalence. Let
	\[\begin{tikzcd}
		E\times_BX\to E\arrow[r,dash,"\simeq"]\arrow[d,two heads]&E\arrow[d,two heads]\\
		X\arrow[r,"\simeq"]&B
	\end{tikzcd}\]
	be the pullback. Then $E\times_BX\to E$ is a homotopy equivalence.
\end{lemma}
\begin{proof}
	Let $g:B\to X$ be the homotopy inverse of $f$.
	
	\textbf{(Step 1)} Construct another pullback
	\[\begin{tikzcd}
		E\times_BB\arrow[r]\arrow[d,two heads]&X\times_BE\arrow[r]\arrow[d,two heads]&E\arrow[d,two heads]\\
		B\arrow[r,"g",swap]&X\arrow[r,"f",swap]&B
	\end{tikzcd}\]
	
	\textbf{(Step 2)} Constuct $E\to E\times_BB$.
	
	Consider
	\[\begin{tikzcd}
		E\arrow[rr,"\id"]\arrow[d]&&E\arrow[d,two heads]\\
		E\times I\arrow[r,"f\times \id",swap]\arrow[rru,dashed]&B\times I\arrow[r]&B?
	\end{tikzcd}\]
	And then $E\to E\times_BB\to E\times_BX\to E$ is homotopic to the identity.
	
	Constructing the other homotopic inverse is the hard part.
	
	\[\begin{tikzcd}
		Z\sqcup Z\arrow[d,swap,"f_1\sqcup f_2"]\arrow[r]\arrow[rdd,bend left]&I\times Z\arrow[ld,dashed]\arrow[d]\arrow[dd,bend left]\\
		E\times_BX\arrow[r]\arrow[d]&E\arrow[d]\\
		X\arrow[r,"\simeq"]&B
	\end{tikzcd}\]
\end{proof}
\begin{coro}
	$B\overset{f}{\to}B$ is homotopy equivalence and $E\to B$ is a fibration, in
	\[\begin{tikzcd}
		E\times_BB\arrow[d]\arrow[r]&E\arrow[d]\\
		B\arrow[r,"f",swap]&B
	\end{tikzcd}\]
	$E\times_BB\to E$ is a homotopy equivalence.
\end{coro}

\begin{exercise}
	If $fg$ is an isomorphism and $f$ and $g$ have right inverses, then $f$ and $g$ are isomorphisms.
\end{exercise}

\begin{lemma}
	Let
	\[\begin{tikzcd}
		A\arrow[d,"g",tail]\arrow[r,"f"]&B\arrow[d]\\
		X\arrow[r]&X\cup_AB
	\end{tikzcd}\]
	be a pushout with $A\to X$ a cofibration. Then the canonical map from the double mapping cylinder $M(f,g)\to X\cup_AB$ is a homotopy equivalence.
\end{lemma}
\begin{remark}
	\[\begin{tikzcd}
		A\arrow[d,"g"]\arrow[r,"f"]&B\\
		X
	\end{tikzcd}\qquad \begin{tikzcd}
		A\arrow[d]\arrow[r,hook]&M_f\arrow[d]\\
		X\arrow[r]&X\cup_AM_f\cong M(f,g)
	\end{tikzcd}\]
\end{remark}

\begin{defn}\leavevmode
	\begin{itemize}
		\item The \textbf{\textit{homotopy pullback}} of a diagram
		\[\begin{tikzcd}
			&Y\arrow[d]\\
			X\arrow[r]&Z
		\end{tikzcd}\]
		is
		\[\begin{tikzcd}
			X\times_{\ev_0}Z^I\times_{\ev_1}Y\arrow[d]\arrow[r]&Y\arrow[d]\\
			X\arrow[r]&Z
		\end{tikzcd}\]
		Intuitively, for any $x\in X$ and $y\in Y$ this object has the space of paths connecting $x$ and $y$.
		
		\item The \textbf{\textit{homotopy fiber}} if $f:Y\to Z$ is the pullback of
		\[\begin{tikzcd}
			&Y\arrow[d,"f"]\\
			pt\arrow[r]&Z
		\end{tikzcd}\]
		$F\subset Z^I\times_ZY\to Z$, where $F$ is the space of paths starting at $x$ and ending at the same point $f(y)$.
	\end{itemize}
\end{defn}
\begin{remark}
	The pullback of
	\[\begin{tikzcd}
		&Z^I\times_ZY\arrow[d]\\
		X\arrow[r]&Z
	\end{tikzcd}\]
	is the motopy pullback of
	\[\begin{tikzcd}
		&Y\arrow[d]\\
		X\arrow[r]&Z
	\end{tikzcd}\]
\end{remark}
\begin{lemma}
	If $X\to Z$ is a fibration then for
	\[\begin{tikzcd}
		&Y\arrow[d]\\
		X\arrow[r,two heads]&Z
	\end{tikzcd}\]
	the map from the pullback to the homotopy pullback is a homotopy equivalence.
\end{lemma}
\begin{proof}
	\[\begin{tikzcd}
		X\times_Z\arrow[r]\arrow[d,"\simeq",dash]&Y\arrow[d,"\simeq"]\\
		X\times_{\ev_0}Z^I\times_{\ev_1}Y\arrow[r,two heads]\arrow[d]&Z^I\times_ZY\arrow[d,two heads]\\
		X\arrow[r]&Z
	\end{tikzcd}\]
\end{proof}

Finally,
\[\begin{tikzcd}
	\hofib f_1\arrow[r]\arrow[d]&\hofib f\arrow[d,two heads]\arrow[r]&*\arrow[d]\\
	*\arrow[r]&X\arrow[r,"f"]&Y
\end{tikzcd}\]
and
\[\begin{tikzcd}
	Z\arrow[dd]\arrow[r]&F(f)\arrow[d]\arrow[dd,bend left,two heads]\\
	&X\times_YY^I\quad\arrow[d]\\
	X\times I\arrow[r]\arrow[uur,dash]&X
\end{tikzcd}\]
and an exact sequence
\[\begin{tikzcd}[column sep=tiny]
	\Omega^2\hofib\arrow[r]&\Omega^2X\arrow[r]&\Omega^2Y\arrow[r]&\Omega\hofib f\arrow[r]&\Omega X\arrow[r]&\Omega Y\arrow[r]&\hofib f\arrow[r]&X\arrow[r,"f"]&Y
\end{tikzcd}\]
\begin{lemma}[Exactness]
	$\forall z$, $[z\hofib f]\to[Z,X]\to[Z,Y]$.
\end{lemma}
and we get the exact sequence
\[\begin{tikzcd}[column sep=tiny]
	\pi_0(\Omega^2X)\arrow[r]&\pi_0(\Omega^2Y)\arrow[r]&\pi_0(\Omega\hofib f)\arrow[r]&\pi_0(\Omega X)\arrow[r]&\pi_0(\Omega Y)\arrow[r]&\pi_0(\hofib f)\arrow[r]&\pi_0(X)\arrow[r]&\pi_0(Y)
\end{tikzcd}\]
and then
\[[S^0,\Omega^2X]=[\Sigma S^0,\Omega X]=[\Sigma^2 S^0,X]=[S^2,X]=\pi_2(X)\]


\subsection{21 march (Serre fibration long exact sequence)}

We've been talking a lot about Hurewickz fibrations. Let's talk about Serre fibrations. Notice that H. fibration $\implies$ S. fibration. What is the most natural example of a Serre fibration?

\begin{prop}
	Let $E$ be a fiber bundle with fiber $F$. Then $f$ is a Serre fibration.
\end{prop}
\begin{proof}
	What sdoes it mean to be a Serre fibration? It means that
	\[\begin{tikzcd}
		I^n\arrow[r]\arrow[d]&E\arrow[d]\\
		I^{n+1}=I^n\times I\arrow[r]\arrow[ur,dashed]&B
	\end{tikzcd}\]
	So if $\Uc$ is a covering of $B$ such that $f^{-1}U\cong U\times F$. By Lebesgue lemma, there is a $\delta>0$ such that for all $x\in I^{n+1}$, the ball $B(x,\delta)$ lies in some $f^{-1}U$ for some $U$.
	
	Then we subdivide $I^{n+1}$ in smaller cubes of the same size with diameter $<\delta$. So, each the image of each cube lies in some $U\in\Uc$.
	
	Then
	\[\begin{tikzcd}
		I^n\arrow[d]\arrow[r]&F\times U\arrow[d,two heads]\\
		I^{n+1}\arrow[r]\arrow[ur,dashed]&U
	\end{tikzcd}\]
	has a lift for every little square because
	\[\begin{tikzcd}
		X\arrow[r]\arrow[d]&U\arrow[d]\\
		X\times I\arrow[r]\arrow[ur,dashed]&pt
	\end{tikzcd}\]
	is always a fibration {\color{orange}(think about this)} and because pullbacks of fibrations are fibrations:
	\[\begin{tikzcd}
		U\times F\arrow[r,two heads]\arrow[d,two heads]&U\arrow[d,two heads]&\\
		F\arrow[r,two heads]&pt
	\end{tikzcd}\].
	Then we may just add up the squares because
	\[\begin{tikzcd}
		D^n\arrow[d]\\
		D^n\times I
	\end{tikzcd}\]
	and we're done.
\end{proof}
\begin{prop}[Construction of homotopy long exact sequence from relative homotopy long exact sequence]
	Let $g:E\to B$ is a Serre fibration. $e\in E$, $g(e)= b$ and $g^{-1}=F$. Then consider the exact sequence in homotopy of the Serre fibration and the relative homotopy exact sequence. Then there is a long exact sequence (top row):
	\[\begin{tikzcd}
		\cdots\arrow[r]&\pi_n(F)\arrow[r]&\pi_n(E)\arrow[r]&\pi_n(B)\arrow[r]&\pi_{n-1}(F)\arrow[r]&\pi_{n-1}(E)\arrow[r]&\cdots\\
		\cdots\arrow[r]&\pi_n(F)\arrow[r]\arrow[u,"="]&\pi_n(E)\arrow[r]\arrow[u,"="]&\pi_n(E,F)\arrow[r]\arrow[u,"\cong",dashed]&\pi_{n-1}(F)\arrow[r]\arrow[u,"="]&\pi_{n-1}(E)\arrow[r]\arrow[u,"="]&\cdots
	\end{tikzcd}\]
\end{prop}
\begin{example}
	We have shown that $\pi_2(\C P^n)\cong\Z$ using the Hopf fibration $S^1\hookrightarrow S^{2n+1}\to\C P^n$ and the fact that $\pi_k(S^n)=0$ for $k<n$.
\end{example}
\begin{thm}
	Let $X$ be a CW-comples, $A,B\subset X$ subcomplexes, $C=A\cap B\neq\varnothing$, so
	\[\begin{tikzcd}
		C\arrow[r]\arrow[d]&A\arrow[d]\\
		B\arrow[r]&X \arrow[ul, phantom, "\ulcorner", very near start]
	\end{tikzcd}\]
	is a pushout (this happens for inclusions, {\color{orange} check it?}).
	
	If $(A,C)$ is $n$-connected and $(B,C)$ is $m$-connected, then
	\[\pi_i(A,C)\to\pi_i(X,B)\]
	is an isomorphism for $i<m+n$ and sujerctive for $i=m+n$.
\end{thm}

\subsection{26 march (Blakers-Massey)}
First I show some basic constructions from Tom Dieck (sec. 5.7). Let $f:X\to Y$ be a map. Consider the pullback
\[\begin{tikzcd}
	W(f)\arrow[r]\arrow[d,"(q\text{,}p)",swap]&Y^I\arrow[d,"(\ev_0\text{,}\ev_1)"]\\
	X\times Y\arrow[r,"f\times\id",swap]&Y\times Y
\end{tikzcd}\]
where
\begin{gather*}
	W(f)=\{(x,w)\in X\times Y^I|f(x)=w(0)\},\\
	q(x,w)=x,\quad p(x,w)=w(1).
\end{gather*}
Since $(\ev_0,\ev_1)$ is a fibration, the maps $(q,p)$, $q$ and $p$ are fibrations.

Now suppose $f$ is a pointed map with base points $*$. Then $W(f)\to W'$ is given the base point $(*,k_*)$.

Let $f:A\hookrightarrow X$ be an inclusion.

\begin{defn}
	By $(I^n,\partial I^n)\to (*\times_{\ev_0}X^I\times_{\ev_1}A,\pt)$ is the same as a map $I^n\times I\to X$ that satisfies:
	\begin{itemize}
		\item $I^n\{0\}\cup\partial I^n\times I\to*$.
		\item $I^n\times\{1\}\to A$.
	\end{itemize}
\end{defn}
It is fairly straightforward to show that
\[\begin{tikzcd}
	\cdots\arrow[r]&\Omega A\arrow[r]&\Omega X\arrow[r]&\hofib\arrow[r]&A\arrow[r]&X
\end{tikzcd}\]
\[\begin{tikzcd}
	\pi_0(\nearrow)=&\pi_n(A)\arrow[r]&\pi_n(X)\arrow[r]\arrow[dr]&\pi_{n-1}(\hofib)\arrow[d,"\cong"]\arrow[r]&\pi_{n-1}(A)\arrow[r]&\pi_{n-1}(X)\\
	&&&\pi_n(X,A)\arrow[ur]
\end{tikzcd}\]
\begin{thm}[Blakers-Massey 1]
	Let 
	\[\begin{tikzcd}
		Q\arrow[r,"g"]\arrow[d,swap,"f"]&Y\arrow[d]\\
		X\arrow[r]&P
	\end{tikzcd}\]
	be a homotopy pushout, $g$ is $m$ equivalence, $f$ is $n$-equivalence and $m,n\geq0$. Then $Q\to X\times_P^h$ is $(m+n-1)$-equivalence.
\end{thm}
\begin{thm}[Blakers-Massey 2]
	$P$ is a CW-complex, $X$, $Y$ subcomplexes, $X\cap Y=Q\neq\varnothing$ (\textbf{\textit{strict pushout}})
	\[\begin{tikzcd}
		Q\arrow[r,tail]\arrow[d,tail]&Y\arrow[d,tail]\\
		X\arrow[r,tail]&X\arrow[ul, phantom, "\lrcorner", very near start]
	\end{tikzcd}\]
	Then $\pi_i(Y,\Q)\to\pi_i(P,X)$ is epi for $i=m+n$ and iso for $0\leq i<m+n$.
\end{thm}
\begin{thm}[Blakers-Massey 3]
	$P=X\cup Y$, $X$ and $Y$ are open in $P$, $X\cap Y=Q\neq\varnothing$.
\end{thm}
We proved the third version based on Tom Dieck's proof.
\begin{defn}\leavevmode
	\begin{itemize}
		\item 	A map is a \textbf{\textit{$k$-equivalence}} if the induced map on the $i$th homotopy group is an isomorphism for $i<k$ and an epimorphism for $i=k$.
		\item $K_p(W):=\{x\in W:\text{ at least $p$ coordinates of $x$ are < the same coordinates of the center of }W\}$
	\end{itemize}
\end{defn}
\begin{lemma}
	Let $W$ be a cube in $\R^d$ with $\dim W\leq d$. If for all faces $W'$ of $\partial W$, $f(W')\in A\implies w'\in K_p(W')$, then there is a homotopy $f\simeq g\;\rel\partial W$ such that $g(w)\in A\implies w\in K_p(W)$.
\end{lemma}
%		B\arrow[r]&X \arrow[ul, phantom, "\lrcorner", very near start]
%		B\arrow[r]&X \arrow[ul, phantom, "\ulcorner", very near start]

\addcontentsline{toc}{section}{References}
\printbibliography
\end{document}